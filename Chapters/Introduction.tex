\chapter{INTRODUCTION}   \label{Introduction}

Power line communication (PLC\acused{PLC}) systems have been investigated by academic and industry sectors worldwide over the past decades. Recently, these systems have become attractive solutions for data communication once they can be deployed over the existing electric power systems infrastructure. As a matter of fact, the ubiquitousness of electric power systems is the main advantage associated with the use of \ac{PLC} technologies. Based on this very important characteristic, \ac{PLC} systems can assist the deployment of \ac{SG}, the \ac{IoT}, smart city, and Industry $4.0$ technologies. On the other hand, one may say that the wireless communications constitute a more established alternative for this purpose due to its low-cost and suitability for fulfilling the needs and demands of these technologies. However, it is widely recognized that \ac{SG}, \ac{IoT}, smart city and Industry $4.0$ will be supported by a heterogeneous set of telecommunication technologies, as no single solution fits all scenarios \cite{Galli2011,Verma}. For the given reasons, \ac{PLC} and unlicensed wireless communications are considered as the two leading data communication technologies for \ac{SG} applications and \ac{IoT} \cite{Sayed2014, Dib, Sayed2015}. 

Some researches have pointed out that the deployment and operational costs related to the \ac{PLC} systems can be low \cite{Hrasnica:PLC_design, Dib} and it constitutes a relevant advantage in favor of these data communication systems; however, electric power grids were originally conceived and designed for generating, transmitting and delivering or distributing energy to consumers or prosumers and, as a consequence, the propagation of signals carrying information through power lines may suffer severe attenuation and/or frequency-selective fading due to the use of non-ideal and unshielded conductors, the existence of impedance mismatching, and dynamics of loads and equipments connected to electric power grids. Also, it is important to highlight that signals carrying information over power line can be corrupted by the high power impulsive noise presences associated with the dynamics of electric power systems and the use of electromagnetically unshielded power lines. In addition, the diversity of topologies and distances of electric power grids may result in very disparate behaviors that demand powerful \ac{PLC} systems for dealing with this problems; power lines work as antennas and, as consequence, interference with other telecommunication systems operating in the same frequency band may result in significant \ac{RF} interference; and the remarkable increase of connections of new types of loads make the electric power grids a challenging media for data communication purposes. Moreover, it is important to pay attention to the voltage level (low, medium and high-voltage), the type of environment, such as indoor (vehicle, residences and building) and outdoor (metropolitan and rural areas), among other issues.

As a common sense in the telecommunication field, it is well-established that the measurement, characterization and modeling of \ac{PLC} channels are mandatory tasks to be \textit{a priori} accomplished for fostering advances in \ac{PLC} systems because they yield important information for driving the designers of telecommunication devices. Due to the complexity and diversity of electric power systems, it is well-accepted by the \ac{PLC} community that \ac{PLC} channels can be, in terms of voltage level, organized in the low-voltage (indoor or outdoor) \cite{Zhai:low}, medium-voltage (outdoor and underground) \cite{Lazaropoulos} and high-voltage (outdoor) \cite{Zajc}. It is important to emphasize that this organization is aligned with the electric power systems field. Also, it deserves attention the fact that the underground medium-voltage electric power grids are more distinct because of the surrounding environment \cite{Aquilue}. 

Focusing on the electric power grids located in indoor facilities, research efforts toward the characterization and modeling of \ac{PLC} channels may be organized in terms of the type of facilities as follows: residential or commercial buildings, known as in-home or in-building \ac{PLC} \cite{Amirshahi:PLC,Tlich:Indoor} and in-vehicles (i.e., cars \cite{Vallejo:Vehicle_PLC}, ships \cite{Barmada:Ships_PLC}, and aircrafts \cite{Jones:Aircraft_PLC,Andrei:Meas,Andrey2016}). Regarding the frequency band of operation, the \ac{PLC} community does not follow the communications community once the concept of the narrowband-\ac{PLC} systems is associated with data communication through the frequency band delimited by $0$ and $500$~kHz \cite{Gassara:Charac_PLC,Chrysochos:MIMO_OFDM}, while the concept of the broadband-\ac{PLC} systems refers to the use of the frequency band from $1.705$ up to $100$~MHz \cite{Tlich:Indoor,Galli:Wireline}, depending on the telecommunications regulation, for data communication purpose. For instance, in Brazil, broadband-\ac{PLC} systems are allowed to operate in the frequency band delimited by $1.7$ and $50$~MHz \cite{Anatel:PLC}. Some contributions point out that the frequency band between $0$ and $500$~MHz may be used by future generation of \ac{PLC} systems \cite{Luis:doc,zeddam1}. In this context, several improvements are being investigated, such as those obtained by cooperative communication \cite{mateus:2018,Michelle2016,Michele:mt,Valencia2014, Roberto2015}. 

Still addressing the indoor electric power grids and focusing on the in-home environments, several contributions related to research efforts for \ac{PLC} channels characterizations carried out in distinct countries can be highlighted. For instance, \cite{Canete:Model} discussed in-home \ac{PLC} channels in Spain that are related to channel attenuation and additive noise in the frequency band between $1.705$ and $30$~MHz, while \cite{Canete:PLC} considered other features, such as delay spread, \ac{ACG} and \ac{CB}. Also, \cite{Cortes:PLC} discussed the normal/log-normal nature of delay spread and \ac{ACG}. In \cite{GalliUS,Galli:Wireline}, the in-home \ac{PLC} channels in some urban and suburban United States (US) residences were characterized in terms of \ac{ACG} and \ac{RMS-DS}, considering the frequency band ranging from $2$ up to $30$~MHz. In \cite{Tlich:Indoor}, a characterization of in-home \ac{PLC} channels was performed in some urban and suburban France residences, describing values of \ac{CB} and time-delay parameters, considering the frequency band ranging from $30$~kHz up to $100$~MHz. Recently, \cite{Thiago:Characterization} has analyzed the \ac{CFR}, \ac{ACA}, \ac{RMS-DS}, \ac{CB}, and \ac{CT} of Brazilian in-home \ac{PLC} channels when the frequency band ranges from $1.7$ up to $100$~MHz. Analyzing the aforementioned works and contributions, we notice that several works had adopted as true that the Log-normal distribution for modeling the \ac{CFR} magnitude; however, the lack of confidence on it \cite{Cortes:PLC} brings our attention to the necessity of further investigation of \ac{CFR} of in-home \ac{PLC} channels.

Concerning \ac{WLC}, it is important to point out the high dependence on line-of-sight propagation, the increasing signal attenuation along with both distance and carrier frequency, the susceptibility to interference among two or more telecommunication systems operating in the same frequency band, the scarcity of spectrum, and vulnerability to non-authorized access, among other things, constitute a relevant set of problems. Also, the transmitted signal through the air suffers three different propagation effects: reflection, scattering and diffraction \cite{Guze}. Remarkable problems in \ac{WLC} are associatated with the distortions introduced by the aforementioned propagation effects and co-channel interference, which is primarily generated from uncoordinated transmissions \cite{Sayed2015}.  

\ac{WLC} channel characterization and modeling has been atracting high attention since they can highly reduce the time of developing \ac{WLC} systems. The \ac{WLC} system support several different technologies and includes, among others, the \ac{WLAN}, comprising five distinct frequency bands: $900$~MHz, $2.4$~GHz, $3.6$~GHz, $5$~GHz, and $60$~GHz \cite{IEEEWI}; the cellular data service, using the main frequency range around $900$~MHz, $1.8$~GHz and $1.9$~GHz \cite{GSM}; the \ac{LPWAN}, which bridges the frequency gap between \ac{WLAN} and cellular technologies, commonly used on \ac{IoT} applications \cite{Patel}. As the number of users and service stations has significantly increased in the last decades, more research efforts to correctly characterize and model \ac{WLC} channel have been carried out. Channel characterization and modeling in such situations can give valuable insights such as predicting the communication quality of a \ac{WLC} system when high data traffic and or large number of users are demanding the same frequency bandwidth. Among the well established \ac{WLC} channel models in the literature we can mention: the COST 2100 channel model \cite{COST}, that provides statically close-to-true descriptions of wireless channels both for indoor scenarios, regarding $3.6$ and $5.3$~GHz frequency bands, and outdoor scenarios, covering the $400$~MHz frequency band; the IEEE 802.15.4a channel model \cite{802.15}, concerning models for diverse frequency ranges and environments, such as indoor and outdoor \ac{UWB} channels covering the frequency range from $2$ up to $6$ GHz; the Hiperlan$/2$ channel model \cite{Hiperlan2}, covering wireless models in different indoor scenarios for the frequency band of $5$~GHz. Regarding the single-input single-output modeling of \ac{WLC} channels, it is worth stating that the frequency band up to $6$~GHz has been very well investigated for characterization and modeling purpose, while the research efforts related related to multiple-input multiple-output is somehow well-established. Currently, a great deal of attention is toward the frequency band related to $5$~GHz and beyond.

In a nutshell, it is important to emphasize that \ac{PLC} and \ac{WLC} media present distinct characteristics that can be jointly exploited for improving the telecommunication systems performance in terms of reliability, data-rate, physical layer security, coverage, and flexibility in the physical and link layers \cite{Victor2018,Dib,Victor2017,Lai2012,Sayed2015,Holden2011}. In this regard, a hybrid approach exploiting the existing diversity between \ac{PLC} and \ac{WLC} communication systems was initially introduced in order to overcome the problems experienced in both isolated systems. Essentially, it assumes that the \ac{PLC} and \ac{WLC} channels are concatenated in series such that \ac{PLC} and \ac{WLC} devices operate in the same frequency band. This type of hybrid data communication system emerged at the very beginning of the XX Century \cite{Mischa} but the advances related to \ac{WLC} system during the following decades drastically reduced the interests in this kind of data communication. Only at the beginning of the XXI Century, precisely one century after its initial investigation, \cite{thiago:hyb} revisited and investigated this kind of channels. As it was initially proposed, it made use of unshielded power lines to propagate and irradiate the transmitted signal by a \ac{PLC} device. In other words, it assumes that the power line acts as an antenna. As a result, the power lines radiate signals and, conversely, wireless signals, generated by a \ac{WLC} device, are inductively injected into the power lines. Currently, there is a great deal of attention in parallel and cascade usages of \ac{PLC} and \ac{WLC} channels to provide either reliability or high data rate. The former approach result in the so-called the hybrid \ac{PLC}/\ac{WLC} system while the later is coined the hybrid \ac{PLC}-\ac{WLC} system. The hybrid \ac{PLC}/\ac{WLC} system have also been investigated as an alternative to improve data communication performance \cite{Victor2016,Victor2017,Victor2018,Lai2012,Sayed2014,Lai2010,Leo2016} since it can take advantage of both electric power grids and air to improve data rate and reliability between source and destination nodes under several circumstances, see \cite{Dib} for details. In such data communication systems, electric power grids and wireless media are used simultaneously in a cooperative perspective to maximize the available channel resources and, as a consequence, to fulfill data communication constraints.

At this time, it is important to emphasize that there are several works for characterizing \ac{PLC} channels but \ac{PLC} channel modeling is still an open problem, once the behavior of electric power grids over narrowband (frequency band from $0$ up to $500$~kHz) and broadband (frequency band from $1.7$ up to $100$~MHz) are different in terms of signal propagation and additive noise influence due to the complexity of electric power systems (e.g., load dynamics, topologies and voltage levels). Therefore, channel modeling is still an open problem related to \ac{PLC} systems. Regarding \ac{WLC} channels, the huge research efforts carried out worldwide have introduced representative models that are well-established in the literature \cite{COST,802.15,Hiperlan2}. Furthermore, when the discussion is toward to hybrid \ac{PLC}-\ac{WLC} channels, which is defined as the series concatenation of \ac{PLC} and \ac{WLC} channels, there is a complete lack of contribution regarding their modeling in the literature. In other words, the investigation of channel models for \ac{PLC} channels and hybrid \ac{PLC}-\ac{WLC} channels is an important research endeavor demanding attention. In this regard, this dissertation strives to cover this gap an to expand the knowledge on these channels by extracting and modeling the benefits of \ac{PLC} and hybrid \ac{PLC}-\ac{WLC} channels for data communication in the frequency band delimited by $1.7$ and $100$~MHz.


%%%%%%%%%%%%%%%%%%%%%%%%%%%%%%%%%%%%%%%%%
\section{OBJECTIVES} \label{sec:I1}
%%%%%%%%%%%%%%%%%%%%%%%%%%%%%%%%%%%%%%%%%

Aiming to pay attention to \ac{PLC} and hybrid \ac{PLC}-\ac{WLC} channels for data communication, this thesis seek to provide statistical characterization and modeling of \acp{CFR} of Brazilian in-home \ac{PLC} and hybrid \ac{PLC}-\ac{WLC} channels, in order to support future efforts for simulating and designing in-home data communication systems. In this context, an enhanced statistical modeling method, which emerged from the drawbacks associated with the methodology presented in \cite{Luis:AI,Luis:doc}, is proposed. Furthermore, the statistical characterization and modeling results, based on data sets of measured \acp{CFR} of in-home \ac{PLC} and hybrid \ac{PLC}-\ac{WLC} channels, are detailed.
The mentioned data sets were acquired during a measurement campaign performed in several Brazilian residences, see more informations about the measurement campaign in \cite{Thiago:Characterization,thiago:hyb,thiago:hyb2,thiago:doc}. In this contribution, the frequency band is set to be from $1.7$ up to $100$~MHz, once it covers broadband \ac{PLC} systems that can offer data rates in the order of $1-2$~Gbps \cite{Galli:indoor,Thiago:Characterization}. Also, it is a frequency band that comprises Brazilian, European and US telecommunication regulations for \ac{PLC} systems. Overall, the main objectives of this thesis may be organized as follows:

\begin{itemize}
	\item To introduce the enhanced statistical modeling method, based on the method presented in \cite{Luis:AI,Luis:doc}, that is capable of searching for the best statistical distribution to model the \ac{CFR} of in-home \ac{PLC} and hybrid \ac{PLC}-\ac{WLC} channels. It treats the magnitude and phase components of \acp{CFR} as random processes. Based on the fact that the proposed enhanced statistical modeling method is designed in the discrete-time domain, it encompasses a procedure for interpolating the parameters values of the chosen statistical distributions, by dividing the desired frequency band into subbands and using the cubic Spline interpolation technique to come up with a concise representation (i.e., finite and low number of parameters) of the parameters of this random process in the continuous-frequency domain. Also, the algorithms that implements the enhanced statistical modeling method is detailed. 
	
	\item To provide statistical analyses of measured Brazilian in-home \ac{PLC} and hybrid \ac{PLC}-\ac{WLC} channels, which were acquired by a measurement campaign carried out several residences and took into account the frequency band between $1.7$ and $100$~MHz. To present the models of such channels and to compare the attained results with the ones found in the literature regarding \ac{PLC} channels. To introduce, for the first time, models of the \acp{CFR} of hybrid \ac{PLC}-\ac{WLC} channels. 
\end{itemize}

%%%%%%%%%%%%%%%%%%%%%%%%%%%%%%%%%%%%%%%%%
\section{THESIS ORGANIZATION} \label{sec:I2}
%%%%%%%%%%%%%%%%%%%%%%%%%%%%%%%%%%%%%%%%%

The remainder of this document is structured as follows: 

\begin{itemize}
	\item Chapter 2 covers the mathematical formulation necessary for dealing with the modeling of \ac{PLC} channels, based on the assumption that the magnitude and phase responses of a \ac{CFR} can be modeled as two random processes in the discrete-time domain. In sequel, the modeling of the \acp{CFR}  of hybrid \ac{PLC}-\ac{WLC} channels is formulated likewise.
	
	\item Chapter 3 describes the proposed enhanced statistical modeling method, based on the method presented in \cite{Luis:AI,Luis:doc}, to obtain the \ac{CFR} model of \ac{PLC} and hybrid \ac{PLC}-\ac{WLC} channels, based on its magnitude and phase responses. The enhanced statistical modeling method also presents a procedure to interpolate the parameters of the statistical distributions that is capable of modeling randomness of magnitude and phase components of \acp{CFR} in the discrete and continuous-time domains.
	
	\item Chapter 4 discusses statistical analyses and numerical results of the proposed enhanced statistical modeling method, over a data set of \acp{CFR} estimates, acquired through a measurement campaign carried out in several Brazilian residences. Also, it offers comparison among the results found in the literature and the ones presented on this thesis.
	 
	\item Chapter 5 summarizes the main contributions and findings achieved in this thesis. Also, it outlines future research endeavors and new problems that came out to attention during the time spent with the investigation of these challenging and interesting channels.
\end{itemize}

%%%%%%%%%%%%%%%%%%%%%%%%%%%%%%%%%%%%%%%%%
\section{SUMMARY} \label{sec:I3}
%%%%%%%%%%%%%%%%%%%%%%%%%%%%%%%%%%%%%%%%%

This chapter has presented a brief introduction of this thesis, addressing important aspects of \ac{PLC} and hybrid \ac{PLC}-\ac{WLC} systems. Also, the main objectives and the organization of this work have been summarized.