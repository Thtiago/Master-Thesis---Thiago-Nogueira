\chapter{CONCLUSIONS} \label{Conclusions}

This work has presented statistical characterizations and modelings of the \ac{CFR} estimates of Brazilian in-home \ac{PLC} and hybrid \ac{PLC}-\ac{WLC} channels, constituting an important result to support future efforts in simulating and designing in-home broadband communication systems. The statistical characterization and modeling have been based on numerical analyses of two data sets of measured \ac{CFR} estimates, covering the frequency band between $1.7$ and $100$~MHz, which were acquired from a measurement campaign performed in seven typical Brazilian residences. 

Chapter \ref{ProblemFormulation} has introduced the problem formulation, in which the main assumption is that the magnitude and phase of the \ac{CFR} estimates are two random processes, with uncorrelated samples, denoted by two \acp{r.v.} ($|{\bf H}|$ and ${\bf \Theta}$, respectively). The adopted road map is based on the fact that these random processes can be individually modeled by statistical distributions and the parameters values of these distribution can be obtained along the frequency domain.  Moreover, some investigation questions were brought to attention regarding the statistical distributions that could model the magnitude and phase processes for the \ac{PLC}, \ac{PLC}-\ac{WLC} \textit{short-path} and \ac{PLC}-\ac{WLC} \textit{long-path} scenarios, separately, and the behavior of such statistical distributions in the frequency band, in other words, if the magnitude and phase random processes could be considered stationary.

Furthermore, Chapter \ref{Proposal} has introduced the enhanced statistical modeling method, which emerged from the drawback found in the method presented in \cite{Luis:AI,Luis:doc}, for modeling the magnitude and phase of \ac{CFR} estimates acquired from a data set. The enhanced statistical modeling method was organized into four steps. The first three steps cover the search for the most suitable statistical distribution among a predetermined set of candidates statistical distributions based on four chosen criteria (\ac{MLE}, \ac{AIC}, \ac{BIC} and \ac{EDC}). The fourth step focuses on the interpolation of the parameters of the chosen statistical distributions, in the frequency domain, through the use of the cubic Spline interpolation technique, resulting in a table of reduced number of polynomials coefficients, for each parameter being interpolated. The enhanced statistical modeling method was designed to generate uncorrelated \ac{CFR} magnitude and phase response. Moreover, Chapter \ref{Proposal} ended by describing two algorithms for implementing the first three steps and the fourth step, respectively. 

Chapter \ref{NumericalResults} has covered the numerical results and statistical analyses. After applying the enhanced statistical modeling method on a data set of \ac{CFR} estimates, acquired through a measurement campaign carried out in several Brazilian residences, the uncorrelated statistical models for the magnitude and phase responses of the \ac{PLC}, hybrid \ac{PLC}-\ac{WLC} \textit{short-path}, and hybrid \ac{PLC}-\ac{WLC} \textit{long-path} scenarios were obtained. The magnitude of the \ac{CFR} estimates has been observed to be a non-stationary random process, modeled by the Beta distribution for the in-home \ac{PLC} channel, a result that disagrees with some previous models adopted by the literature. Moreover, numerical result have shown that the magnitude response of \acp{CFR} for both in-home hybrid \ac{PLC}-\ac{WLC} \textit{short-path} and \textit{long-path} channels are best modeled by the Log-normal distribution. The number of subbands used to generate the interpolated curve of the parameters values, through the cubic spline interpolation, was carefully chosen by adopting a procedure combining a Monte Carlo simulation of \ac{MSE} values for allowing a decision about the  number of subbands to be adopted when computational complexity for obtaining the polynomials is taken into account. Regarding the \ac{PLC} channel scenario, the cubic spline interpolation required a total of nineteen third order polynomials in order to interpolate the parameters of the Beta distribution over the desired frequency band. As to the hybrid \ac{PLC}-\ac{WLC} \textit{short-path} and \textit{long-path} scenarios, the cubic spline interpolation required a total of fifteen third order polynomials, for each case, in order to interpolate the parameters of the Log-normal distributions over the desired frequency band. The polynomials coefficients were presented in six different tables located in the Appendix section. Regarding of the phase response of \acp{CFR}, all three different scenarios presented in this thesis (i.e., \ac{PLC} channel, hybrid \ac{PLC}-{WLC} \textit{short-path} channel and hybrid \ac{PLC}-{WLC} \textit{long-path} channel) have been observed to be stationary random processes,  with samples modeled by the Uniform distribution defined in the interval represented by $[0,2\pi]$, regardless of the frequency value. 

%%%%%%%%%%%%%%%%%%%%%%%%%%%%%%%%%%%%%%%%%%%%%%%%%%%%%%%%%%%%%%%%%%%%%%%
\section{FUTURE WORKS}  \label{FW}
%%%%%%%%%%%%%%%%%%%%%%%%%%%%%%%%%%%%%%%%%%%%%%%%%%%%%%%%%%%%%%%%%%%%%%%
A list of future works is as follows:

\begin{itemize}
	\item To investigate the insertion of correlation among consecutive samples of the \ac{CFR} magnitude and phase components, in order to generate models for \ac{PLC} and hybrid \ac{PLC}-\ac{WLC} channels that can be used to perform data communication through these channels. 
	
	\item To extend the enhanced statistical modeling method to handle channel modeled as cyclostationary random processes.

	\item To apply the enhanced statistical modeling method in other data communication channels (i.e., broadband and narrowband outdoors and narrowband indoor).
\end{itemize}