%%%%%%%%%%%%%%%%%%%%%%%%%%%%%%%%%%%%%%%%%%%%%%%%%%%%%%%%%%%%%%%%%%%%%%%%%%%%%%%%%%%%%%%%%%%%%%%%%%%%%%%%%%%%%%%%%%%%%%%%%%%%%%%%%%%%%%%%%%%%%%%%%%%%%%%%%%%%%%%%%%%%%%%%%%%%%%%%%%%%%%%%%%%%%%%%%%%%%%%%%%%%%%%%%%%%%%%%%%%%%%%%%%%%%%%%%%%%%%%%%%%%%%%%%%%Packages.
\usepackage{lmodern}						
\usepackage[T1]{fontenc}
\usepackage[latin1]{inputenc}
\usepackage{lastpage}			
\usepackage{indentfirst}		
\usepackage{color}				
\usepackage{microtype} 	
\usepackage[printonlyused]{acronym}
\usepackage[table]{xcolor}
%\usepackage[breaklinks=true]{hyperref}
\usepackage{amsmath,amssymb,amsthm,mathrsfs,amsfonts,dsfont}
\usepackage{psfrag}
%\usepackage[boxed,lined,figure]{algorithm2e}
\usepackage{multirow}
\usepackage{cite}
\usepackage{rotating}
\usepackage{ifpdf}
\usepackage{pdflscape}
\usepackage{caption}
\usepackage{url}
%\usepackage{subcaption}
\usepackage{subfig}
\usepackage{graphicx}
%\usepackage{xcolor}
%\usepackage{graphicx,subfigure}
\usepackage{booktabs}
\usepackage{textcomp}
%\usepackage{acronym}
\usepackage{enumitem}
\usepackage{array}
\newcolumntype{C}[1]{>{\centering}m{#1}}
%\usepackage{amsmath}
%\usepackage{amssymb}
\usepackage{color}
%\usepackage{amsfonts}
%\usepackage{psfrag}
\usepackage[justification=centering]{caption}
\hyphenation{op-tical net-works semi-conduc-tor}
\usepackage[algoruled]{algorithm2e}
%\renewcommand{\ss}{\resizebox{4pt}{2pt}{$\square$}}


%%%%%%%%%%%%%%%%%%%%%%%%%%%%%%%%%%%%%%%%%%%%%%%%%%%%%%%%%%%%%%%%%%%%%%%%%%%%%%%%%%%%%%%%%%%%%%%%%%%%%%%%%%%%%%%%%%%%%%%%%%%%%%%%%%%%%%%%%%%%%%%%%%%%%%%%%%%%%%%%%%%%%%%%%%%%%%%%%%%%%%%%%%%%%%%%%%%%%%%%%%%%%%%%%%%%%%%%%%%%%%%%%%%%%%%%%%%%%%%%%%%%%%%%%%%DADOS THESIS.

\titulo{Statistical Modeling of the Brazilian In-Home Channels: PLC and Cascade of PLC and Wireless}
\autor{Thiago Fernandes do Amaral Nogueira}
\autorR{Nogueira, Thiago}
\local{Juiz de Fora}
\data{2018}
\orientador[Orientador:]{Mois\'{e}s Vidal Ribeiro}
\coorientador[Coorientador:]{Guilherme Ribeiro Colen}
\instituicao{Universidade Federal de Juiz de Fora}
\faculdade{Engenharia El\'etrica}
\programa{Programa de P\'{o}s-Gradua\c{c}\~{a}o em Engenharia El\'{e}trica}
\objeto{Disserta\c{c}\~{a}o de Mestrado}  %%Tese (Doutorado)
\natureza{Disserta\c{c}\~{a}o de mestrado apresentada ao \insereprograma~da \insereinstituicao, na \'{a}rea de concentra\c{c}\~{a}o em sistemas eletr\^{o}nicos, como requisito parcial para obten\c{c}\~{a}o do t\'{i}tulo de Mestre em Engenharia El\'{e}trica.}
% Abaixo, prencher com os dados da parte final da ficha catalográfica.
\finalcatalog{1. Caracteriza\c{c}\~ao estat\'istica. 2. Comunica\c{c}\~ao h\'ibrida. 3.  Comunica\c{c}\~ao via rede de energia el\'etrica. 4. Comunica\c{c}\~ao sem fio. 5. Metodologia de modelagem. I. Vidal Ribeiro, Mois\'{e}s, orient. II. Ribeiro Colen, Guilherme, coorient. III. T\'{i}tulo.}
%% Aqui fica escrito a palavra ``T\'itulo'' mesmo, nao o do trabalho. Se tiver coorientador, os dados ficam depois dos dados do orientador (II. Sobrenome, Nome do coorientador, coorient.) e antes de ``II. T\'itulo'', o qual passa a ``III. T\'itulo''.
%%%%%%%%%%%%%%%%%%%%%%%%%%%%%%%%%%%%%%%%%%%%%%%%%%%%%%%%%%%%%%%%%%%%%%%%%%%%%%%%%%%%%%%%%%%%%%%%%%%%%%%%%%%%%%%%%%%%%%%%%%%%%%%%%%%%%%%%%%%%%%%%%%%%%%%%%%%%%%%%%%%%%%%%%%%%%%%%%%%%%%%%%%%%%%%%%%%%%%%%%%%%%%%%%%%%%%%%%%%%%%%%%%%%%%%%%%%%%%%%%%%%%%%%%%%NewAcronyms.
\newacro{HIPERLAN}{high-performance radio local area network}
\newacro{HSRC}{hybrid single-relay channel}
\newacro{HOHC}{hybrid one-hop channel}
\newacro{PLC}{power line communication}
\newacro{SRC}{single-relay channel}
\newacro{2S-SRC}{two-stage SRC}
\newacro{MIMO}{multiple-input multiple-output}
\newacro{AF}{amplify-and-forward}
\newacro{DF}{decode-and-forward}
\newacro{TDMA}{time-division multiple access}
\newacro{MRC}{maximum ratio combining}
\newacro{LGRC}{linear Gaussian relay channel}
\newacro{$N$-CGRC}{$N$-block circular Gaussian relay channel}
\newacro{OFDM}{orthogonal frequency-division multiplexing}
\newacro{HS-OFDM}{hermitian-symmetric orthogonal frequency-division multiplexing}
\newacro{PSD}{power spectral density}
\newacro{BER}{bit error rate}
\newacro{nSNR}{normalized signal-to-noise ratio}
\newacroindefinite{nSNR}{an}{a}
\newacro{CFR}{channel frequency response}
\newacro{LTV}{linear time-variant}
\newacro{LTI}{linear time-invariant}
\newacroindefinite{LTI}{an}{a}
\newacro{DFT}{discrete Fourier transform}
\newacro{AWGN}{additive white Gaussian noise}
\newacro{SNR}{signal-to-noise ratio}
\newacroindefinite{SNR}{an}{a}
\newacro{CIR}{channel impulse responses}
\newacro{CSI}{channel state information}
\newacro{LAN}{Local Area Network}
\newacro{ETSI}{European Telecommunications Standards Institute}
\newacro{NLOS}{non-line-of-sight}
\newacro{BRAN}{Broadband Radio Access Networks}

\newacro{ACG}{average channel gain}
\newacro{CB}{coherence bandwidth}
\newacro{ACA}{average channel attenuation}
\newacro{RMS-DS}{root mean squared delay spread}
\newacro{CT}{coherence time}
\newacro{DFT}{discrete Fourier transform}
\newacro{BPSK}{binary phase shift keying}
\newacro{r.va.}{random variables}
\newacro{CP}{cyclic prefix}
\newacro{SFO}{sampling-frequency offset}
\newacro{MLE}{Maximum Likelihood Estimation}
\newacro{AIC}{Akaike information criterion}
\newacro{BIC}{Bayesian information criterion}
\newacro{EDC}{Efficient determination criterion}
\newacro{LPTV}{linear periodically time varying}
\newacro{IFT}{inverse Fourier transform}
\newacro{r.v.}{random vector}
\newacro{FM}{frequency modulation}
\newacro{MSE}{mean squared error}
\newacro{RMSE}{root-mean-square error}
\newacro{WLC}{wireless communication} 
\newacro{IoT}{Internet of Things}
\newacro{RF}{radio frequency}
\newacro{SG}{smart grid}
\newacro{WLAN}{wireless local area network} 
\newacro{LPWAN}{Low Power Wide Area Networks}
\newacro{HiperLAN}{High Performance Radio LAN}
\newacro{UWB}{Ultra-wide-band}
%%%%%%%%%%%%%%%%%%%%%%%%%%%%%%%%%%%%%%%%%%%%%%%%%%%%%%%%%%%%%%%%%%%%%%%%%%%%%%%%%%%%%%%%%%%%%%%%%%%%%%%%%%%%%%%%%%%%%%%%%%%%%%%%%%%%%%%%%%%%%%%%%%%%%%%%%%%%%%%%%%%%%%%%%%%%%%%%%%%%%%%%%%%%%%%%%%%%%%%%%%%%%%%%%%%%%%%%%%%%%%%%%%%%%%%%%%%%%%%%%%%%%%%%%%%Others.
%%%%%%%%%%%%%%%%%%%%%%%%%%%%%%%%%%%%%%%%%%%%%%%%%%%%%%%%%%%%%%%%%%%%%%%%%%%%%%%%%%%%%%%%%%%%%%%%%%%%%%%%%%%%%%%%%%%%%%%%%%%%%
\setlength{\parindent}{1.3cm}
\setlength{\parskip}{0.2cm}  
\setlength\afterchapskip{12pt}
%%%%%%%%%%%%%%%%%%%%%%%%%%%%%%%%%%%%%%%%%%%%%%%%%%%%%%%%%%%%%%%%%%%%%%%%%%%%%%%%%%%%%%%%%%%%%%%%%%%%%%%%%%%%%%%%%%%%%%%%%%%%%
% FIGURES.
\graphicspath{{./Figs/}}
\DeclareGraphicsExtensions{.eps}
%%%%%%%%%%%%%%%%%%%%%%%%%%%%%%%%%%%%%%%%%%%%%%%%%%%%%%%%%%%%%%%%%%%%%%%%%%%%%%%%%%%%%%%%%%%%%%%%%%%%%%%%%%%%%%%%%%%%%%%%%%%%%
% NEW THEOREM
\newtheoremstyle{note}% <name>
{3pt}% <Space above>
{3pt}% <Space below>
{\itshape}% <Body font>
{}% <Indent amount>
{\itshape}% <Theorem head font>
{:}% <Punctuation after theorem head>
{.5em}% <Space after theorem headi>
{}% <Theorem head spec (can be left empty, meaning `normal')>
\theoremstyle{note}
\newtheorem{defn}{Definition}
%%%%%%%%%%%%%%%%%%%%%%%%%%%%%%%%%%%%%%%%%%%%%%%%%%%%%%%%%%%%%%%%%%%%%%%%%%%%%%%%%%%%%%%%%%%%%%%%%%%%%%%%%%%%%%%%%%%%%%%%%%%%%
\DeclareOldFontCommand{\bf}{\normalfont\bfseries}{\mathbf}
\DeclareOldFontCommand{\rm}{\normalfont\rmfamily}{\mathrm}
%%%%%%%%%%%%%%%%%%%%%%%%%%%%%%%%%%%%%%%%%%%%%%%%%%%%%%%%%%%%%%%%%%%%%%%%%%%%%%%%%%%%%%%%%%%%%%%%%%%%%%%%%%%%%%%%%%%%%%%%%%%%%
% NEW COMMANDS
\newcommand{\spacing}{0.6}
\newcommand{\mat}[1]{\mbox{\boldmath{$#1$}}} 

\newcommand{\M}{{\Large x}}
\newcommand{\Atv}[1]{\multicolumn{1}{|l|}{Task \##1}}
\newcommand{\diag}{\mathop{\mathrm{{\bf diag}}}}
\newcommand{\card}{\mathop{\mathrm{card}}}
\newcommand{\Tr}{\mathop{\mathrm{Tr}}}
\newcommand{\Largura}{15cm}
\newcommand{\Larguraa}{14cm}
\renewcommand{\a}{a}
\renewcommand{\b}{b}
\newcommand{\HS}{\mathop{\text{{\tiny hs}}}}
\renewcommand{\th}{\mathop{^{th}}}

\newcolumntype{C}[1]{>{\centering\let\newline\\\arraybackslash\hspace{0pt}}m{#1}}

\makeatletter% \tiny: 6/7
\newcommand{\mm}{\@setfontsize\miniscule{5.5}{6.5}}
\makeatother

\newcommand{\Zdiag}{\mathop{\mathrm{\mathcal{Z}}}}
\newcommand{\Eq}{\mathop{\mathrm{Eq}}}
\newcommand{\destacar}[1]{\textcolor{red}{#1}}
\newcommand{\cor}[1]{\textcolor{red}{#1}}
%%%%%%%%%%%%%%%%%%%%%%%%%%%%%%%%%%%%%%%%%%%%%%%%%%%%%%%%%%%%%%%%%%%%%%%%%%%%%%%%%%%%%%%%%%%%%%%%%%%%%%%%%%%%%%%%%%%%%%%%%%%%%
\hyphenation{through-put band-width ti-me--fre-quen-cy}
\newacro{BRAN}{Broadband Radio Access Networks}

